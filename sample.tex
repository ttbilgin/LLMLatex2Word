In this study, $\mathbf{XGBoost-based\:imputation}$ was implemented through a systematic machine learning approach. The methodology employs a regression model tailored for handling missing values in the air quality dataset. The imputation process can be mathematically formulated as follows:

Given a set of meteorological features
\[
\mathbf{X} = \{x_1, \dots, x_n\}, \quad \text{where} \, x_i \in \mathbb{R}^d
\]
Here, \( x_i \) represents the \( d \)-dimensional input features (e.g., blh, msl, TdegC, rh, w_s), and corresponding PM2.5 values
\[
\mathbf{Y} = \{y_1, \dots, y_n\}.
\]

The \textbf{XGBoost model} predicts missing values through an additive function:
\[
\hat{y}_i = \phi(\mathbf{x}_i) = \sum_{k=1}^K f_k(\mathbf{x}_i), \quad f_k \in \mathcal{F}
\]
where \( \mathcal{F} \) represents the space of regression trees, \( K = 100 \) is the number of trees, and \( f_k \) denotes the \( k \)-th tree in the ensemble.

The model is trained by minimizing the \textbf{regularized objective function}:
\[
\mathcal{L} = \sum_{i=1}^n (\hat{y}_i - y_i)^2 + \sum_{k=1}^K \Omega(f_k),
\]
where the \textbf{regularization term} \( \Omega(f_k) \) is defined as:
\[
\Omega(f_k) = \gamma T + \frac{1}{2}\lambda \|\mathbf{w}\|^2.
\]

Here:
- \( T \): Number of leaves in the tree,
- \( \gamma \): Minimum loss reduction required to make a split (\( \gamma = 0 \) in this case),
- \( \lambda \): L2 regularization term (\( \lambda = 1 \)),
- \( \mathbf{w} \): Leaf weights, and \( \|\mathbf{w}\|^2 \) represents the L2 norm of the leaf weights.

The model utilizes a \textbf{subsample ratio} of 0.8 for both instances and features to prevent overfitting:
\[
\mathbf{X}_{sub} = \text{sample}(\mathbf{X}, \eta=0.8),
\]
where \( \eta \) represents the sampling ratio.

The \textbf{prediction error} is quantified using the \textbf{Root Mean Squared Error (RMSE)}:
\[
RMSE = \sqrt{\frac{1}{n} \sum_{i=1}^n (\hat{y}_i - y_i)^2}.
\]

This mathematical framework enables the model to capture complex relationships between meteorological parameters and PM2.5 concentrations, facilitating \textbf{accurate imputation} of missing values in the air quality dataset.




